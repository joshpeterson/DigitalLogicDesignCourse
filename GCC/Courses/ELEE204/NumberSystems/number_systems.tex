\title{Number Systems}
\begin{document}

\section{Positional Number Systems}

\subsection{Decimal Numbers}

\begin{frame}{Everything you ever needed to know you learned in kindergarten}
  \begin{definition}
    In a \alert{positional number system} each number is represented by a string of digits.  The position of each digit has an associated weight.  The value of the number is the weighted sum of the digits.
  \end{definition}
  \begin{example}
    $8675309 = 8\cdot10^6+6\cdot10^5+7\cdot10^4+5\cdot10^3+3\cdot10^2+0\cdot10^1+9\cdot10^0$
    $3.1415 = 3\cdot10^0+1\cdot10^{-1}+4\cdot10^{-2}+1\cdot10^{-3}+5\cdot10^{-4}$
  \end{example}
  \begin{definition}
    The \alert{base} or \alert{radix} of a number system is the weight of the second digit to the left of the point.  In the decimal number system, the radix($r$) is 10.
  \end{definition}
\end{frame}

\begin{frame}{Generalization of decimal number system}
  Suppose we have a decimal number $D$ such that $D$ has $p$ digits to the left of the decimal point and $n$ digits to the right of the decimal point, then $D$ can be written as
  $$d_{p-1} d_{p-2} \cdots d_1 d_0 . d_{-1} d_{-2} \cdots d_{-n}$$
  Then the value of $D$ is given by
  $$D=\sum_{i=-n}^{p-1}d_i \cdot r^i = \sum_{i=-n}^{p-1}d_i \cdot 10^i$$
\end{frame}

\begin{frame}[t]{Decimal generalization example}
  What is the circumference of a circle with diameter $8675309$ mm?
  $$C = \pi \cdot d = 3.14 \cdot 8675309 \text{ mm} = 27240470.26 \text{ mm} \approx 27.2404 \text{ m}$$
  $$C = \sum_{i=-n}^{p-1}d_i \cdot r^i = \sum_{i=-4}^{1}d_i \cdot 10^i, \text{ with } p=2, n=4$$
  \begin{columns}
    \begin{column}{2cm}
      \begin{tabular}{c|cl}
        $i$ & $d_i$ & $10^i$ \\
        \hline
        -4 & 4 & 0.0001 \\
        -3 & 0 & 0.001 \\
        -2 & 4 & 0.01 \\
        -1 & 2 & 0.1 \\
        0 & 7 & 1 \\
        1 & 2 & 10 \\
      \end{tabular}
    \end{column}
    \begin{column}{7cm}
      $C \approx 27.2404 = 2 \cdot 10 + 7 \cdot 1 + 2 \cdot 0.1 + 4 \cdot 0.01 + 0 \cdot 0.001 + 4 \cdot 0.0001$
    \end{column}
  \end{columns}
\end{frame}

\subsection{Binary Numbers}

\begin{frame}{Binary number system}
  The value of a binary number $b_{p-1} b_{p-2} \cdots b_1 b_0 . b_{-1} b_{-2} \cdots b_{-n}$ is
  $$B=\sum_{i=-n}^{p-1}b_i \cdot r^i = \sum_{i=-n}^{p-1}b_i \cdot 2^i$$
  \begin{example}
    \begin{tabular}{llllll}
      $11010_2$ & $= 1 \cdot 2^4$ & $+ 1 \cdot 2^3$ & $+ 0 \cdot 2^2$ & $+ 1 \cdot 2^1$ & $+ 0 \cdot 2^0$\\
      & $= 1 \cdot 16$ & $+ 1 \cdot 8$ & $+ 0 \cdot 4$ & $+ 1 \cdot 2$ & $+ 0 \cdot 1$\\
    \end{tabular}
  \end{example}
  \begin{definition}
    A \alert{bit} is a single binary digit.  The leftmost bit of a binary number is the \alert{high-order bit} or \alert{most significant bit (MSB)}.  The rightmost bit is the \alert{low-order bit} or \alert{least significant bit (LSB)}.
  \end{definition}
\end{frame}

\begin{itemize}
  \item Recall the analog to digital simplification.  The analog input signal was given two values, 0 and 1.  To make sense of these values, we'll need a number system with radix 2, the binary number system.
  \item Note that we usually denote the radix of a given number using a subscript, except when the number is decimal.  In this case, the radix is omitted.
  \item Another example:
\end{itemize}

\begin{tabular}{llllll}
  $11.001_2$ & $= 1 \cdot 2^1$ & $+ 1 \cdot 2^0$ & $+ 0 \cdot 2^{-1}$ & $+ 0 \cdot 2^{-2}$ & $+ 1 \cdot 2^{-3}$\\
  & $= 1 \cdot 2$ & $+ 1 \cdot 1$ & $+ 0 \cdot 0.5$ & $+ 0 \cdot 0.25$ & $+ 1 \cdot 0.125$
\end{tabular}

\subsection{Octal and Hexadecimal Numbers}

\begin{frame}{Octal and Hexadecimal number systems}
  Since binary number strings can quickly become very long, we often using octal (radix 8) and hexadecimal (radix 16) number systems.
  \begin{itemize}
    \item A string of three bits can take on $2^3 = 8$ values, so one octal digit represents three bits.
    \item A string of four bits can take on $2^4 = 16$ values, so one hexadecimal digit represents four bits.
  \end{itemize}
  \begin{example}
    $110100101001_2 = 110 \text{ } 100 \text{ } 101 \text{ } 001_2= 6451_8$
    $110100101001_2 = 1101 \text{ } 0010 \text{ } 1001_2 = \text{D}29_{16}$
  \end{example}
\end{frame}

\begin{frame}{More on hexadecimal}
  \begin{definition}
    A \alert{byte} is usually and 8-bit (2 hex digit) number.  A 4-bit (1 hex digit) number is called a \alert{nibble}.
  \end{definition}
  \begin{block}{Letters or numbers?}
    In the hexadecimal number system, the decimal numbers 10 - 15 are represented by the letters A - F.
  \end{block}
  \begin{block}{Alternate notation}
    Often, hexadecimal numbers are represented with the prefix 0x, as 0xD29 instead of $\text{D}29_{16}$.
  \end{block}
\end{frame}

\begin{itemize}
  \item The alternate representation is often used because type setting a subscript is not always easy.
  \item Table 2-1 on page 28 of the text contains a table of decimal, binary, octal, and hexadecimal number conversions.  These will become very important.
\end{itemize}

\subsection{Number System Conversions}

\begin{frame}{Convert binary, octal, or hexadecimal to decimal}
  \begin{example}
    Using the following equation, convert each of the numbers to decimal.
    $$D=\sum_{i=-n}^{p-1}d_i \cdot r^i$$
    \\
    $11101_2 =$
    \\
    $4371_8 =$
    \\
    $42A.\text{C}_{16} =$
  \end{example}
\end{frame}

$$11101_2 = 1 \cdot 2^4 + 1 \cdot 2^3 + 1 \cdot 2^2 + 0 \cdot 2^1 + 1 \cdot 2^0 = 1 \cdot 16 + 1 \cdot 8 + 1 \cdot 4 + 0 \cdot 2 + 1 \cdot 1 = 29$$
$$4371_8 = 4 \cdot 8^3 + 3 \cdot 8^2 + 7 \cdot 8^1 + 1 \cdot 8^0 = 4 \cdot 512 + 3 \cdot 64 + 7 \cdot 8 + 1 \cdot 1 = 2048 + 192 + 56 + 1 = 2297$$
$$42A.C_{16} = 4 \cdot 16^2 + 2 \cdot 16^1 + 10 \cdot 16^0 + 12 \cdot 16^{-1} = 4 \cdot 256 + 2 \cdot 16 + 10 \cdot 1 + 12 \cdot 0.0625 = 1024 + 32 + 10 + 0.75 = 1066.75$$

\begin{frame}{Convert decimal to binary, octal, and hexadecimal}
    Successively divide the decimal number by the desired radix (2, 8, or 16).  The remainder of each division is one digit in the converted number, starting with the least significant digit.
  \begin{example}
    Convert 114 to binary, octal, and hexadecimal.\\
    \begin{center}
      \begin{tabular}{c|cc}
        D & Q & r\\
        \hline
        114 & 57 & 0\\
        57 & 28 & 1\\
        28 & 14 & 0\\
        14 & 7 & 0\\
        7 & 3 & 1\\
        3 & 1  & 1\\
        1 & 0 & 1\\
      \end{tabular}
    \end{center}
  \end{example}
\end{frame}

Octal:

\begin{tabular}{c|cc}
  D & Q & r\\
  \hline
  114 & 14 & 2\\
  14 & 1 & 6\\
  1 & 0 & 1\\
\end{tabular}

Hexadecimal:

\begin{tabular}{c|cc}
  D & Q & r\\
  \hline
  114 & 7 & 2\\
  7 & 0 & 7\\
\end{tabular}

\begin{itemize}
  \item Note that conversion between binary, octal, and hexadecimal is simple substitution.
  \item Table 2-2 on page 31 of the text summarizes the conversion rules.
\end{itemize}

\section{Binary Arithmetic}

\subsection{Binary Addition}

\begin{frame}{We didn't learn this in kindergarten}
  \begin{itemize}
    \item Binary addition behaves in the same manner as decimal addition.  However, the addition table is different.
    \item To add numbers $x$ and $y$ we add their values plus a carry in bit ($c_{in}$) to produce a sum ($s$) and a carry out bit ($c_{out}$).
  \end{itemize}
  \begin{center}
      \begin{tabular}{ccc|cc}
        $c_{in}$ & $x$ & $y$ & $s$ & $c_{out}$\\
        \hline
        0 & 0 & 0 & 0 & 0 \\
        0 & 0 & 1 & 1 & 0 \\
        0 & 1 & 0 & 1 & 0 \\
        0 & 1 & 1 & 0 & 1 \\
        1 & 0 & 0 & 1 & 0 \\
        1 & 0 & 1 & 0 & 1 \\
        1 & 1 & 0 & 0 & 1 \\
        1 & 1 & 1 & 1 & 1 \\
      \end{tabular}
  \end{center}
\end{frame}

This table is taken from Table 2-3 on pg 32 of Wakerly.

\begin{frame}{Binary addition example}
  \begin{example}
    Perform the following addition in decimal and binary.
    $$189 + 107$$
  \end{example}
\end{frame}

\begin{tabular}{ccccccccccccccc}
  $C$   &   &   &   &   & & 1 & 1 & 1 & 1 & 1 & 1 & 1 & 1 & 0 \\
  $X$   &   & 1 & 8 & 9 & &   & 1 & 0 & 1 & 1 & 1 & 1 & 0 & 1 \\
  $Y$   & + & 1 & 0 & 7 & & + & 0 & 1 & 1 & 0 & 1 & 0 & 1 & 1 \\
  \hline
  $X+Y$ &   & 2 & 9 & 6 & & 1 & 0 & 0 & 1 & 0 & 1 & 0 & 0 & 0 \\
\end{tabular}

\subsection{Binary Subtraction}

\begin{frame}{Binary subtraction}
  \begin{itemize}
    \item Binary subtraction behaves in the same manner as decimal subtraction.  However, the subtraction table is different.
    \item We subtract $y$ from $x$ and a borrow in bit ($b_{in}$) to produce a difference ($d$) and a borrow out bit ($b_{out}$).
  \end{itemize}
  \begin{center}
      \begin{tabular}{ccc|cc}
        $b_{in}$ & $x$ & $y$ & $d$ & $b_{out}$\\
        \hline
        0 & 0 & 0 & 0 & 0 \\
        0 & 0 & 1 & 1 & 1 \\
        0 & 1 & 0 & 1 & 0 \\
        0 & 1 & 1 & 0 & 0 \\
        1 & 0 & 0 & 1 & 1 \\
        1 & 0 & 1 & 0 & 1 \\
        1 & 1 & 0 & 0 & 0 \\
        1 & 1 & 1 & 1 & 1 \\
      \end{tabular}
  \end{center}
\end{frame}

\begin{frame}{Binary subtraction example}
  \begin{example}
    Perform the following subtraction in decimal and binary.
    $$189 - 107$$
  \end{example}
\end{frame}

\begin{tabular}{ccccccccccccccc}
  $ $   &   &   &   &   & &   &   & 10&   &   &   &   & 10&   \\
  $B$   &   &   &   &   & &   & 1 & 0 & 0 & 0 & 0 & 1 & 0 & 0 \\
  $X$   &   & 1 & 8 & 9 & &   & 1 & 0 & 1 & 1 & 1 & 1 & 0 & 1 \\
  $Y$   & - & 1 & 0 & 7 & & - & 0 & 1 & 1 & 0 & 1 & 0 & 1 & 1 \\
  \hline
  $X-Y$ &   &   & 8 & 2 & &   & 0 & 1 & 0 & 1 & 0 & 0 & 1 & 0 \\
\end{tabular}

\section{Representation of Negative Numbers}

\subsection{Signed Magnitude Representation}

\begin{frame}{Signed magnitude representation}
  \begin{itemize}
    \item In decimal, we used signed magnitude representation.  We place a minus symbol(-) before a number to denote a negative value.
    \item In binary, the MSB is traditional used as a sign bit.  So in an 8-bit number, 7 bits represent the value and one bit represents the sign.
    \item This representation is very readable, but difficult to implement in digital circuits, so it is seldom used.
  \end{itemize}
  \begin{example}
    $$01101101_2 = 109 \text{    } 11101101_2 = -109$$
  \end{example}
\end{frame}

\subsection{Two's Complement Representation}

\begin{frame}{Two's complement representation}
  \begin{definition}
    The \alert{two's complement} (in general, radix complement) of an $n$-bit binary number is obtained by subtracting it from $2^n$ ($r^n$).
  \end{definition}
  \begin{center}
    \begin{tabular}{c|cc}
      Number & Unsigned & 2's Complement \\
      \hline
      $000_2$ & 0 & 0 \\
      $001_2$ & 1 & 1 \\
      $010_2$ & 2 & 2 \\
      $011_2$ & 3 & 3 \\
      $100_2$ & 4 & -4 \\
      $101_2$ & 5 & -3 \\
      $110_2$ & 6 & -2 \\
      $111_2$ & 7 & -1 \\
    \end{tabular}
  \end{center}
\end{frame}

\begin{frame}{Another way to think about two's complement}
  \begin{center}
    \begin{tabular}{cc}
      $000_2$ = 0 & \\
      $001_2$ = 1 & $111_2$ = -1\\
      $010_2$ = 2 & $110_2$ = -2\\
      $011_2$ = 3 & $101_2$ = -3\\
                  & $100_2$ = -4 \\
    \end{tabular}
  \end{center}
\end{frame}

\begin{frame}{One more way to think about the two's complement}
  \begin{itemize}
    \item To find the decimal equivalent of a binary two's complement number use the weighted sum as normal.
    \item However, the weight of the MSB is $-2^{n-1}$.
  \end{itemize}
  \begin{example}
    Find the decimal equivalent of the binary two's complement number $11010001_2$.
    $$1010001_2 = 1 \cdot 64 + 1 \cdot 16 + 1 \cdot 0 = 81$$
    $$11010001_2 = 1 \cdot -128 + 1 \cdot 64 + 1 \cdot 16 + 1 \cdot 0 = -47$$
  \end{example}
\end{frame}

\begin{itemize}
  \item Note that this is very different from the signed magnitude representation.
  \item But why would we do this, it seems much more complex?  We'll find out in a moment.
\end{itemize}

\begin{frame}{Finding the two's complement}
  \begin{block}{Two's Complement Conversion}
    To find the two's complement of a binary number, complement all of the bits (change 0 to 1 and 1 to 0), then add 1 to the result.
  \end{block}
  \begin{example}
    Find the eighth bit two's complement representation of -81.
    \begin{tabular}{rl}
      $81 =$ & $01010001_2$\\
             & $\Downarrow \text{complement}$\\
             & $10101110_2$\\
         $+$ & $00000001_2$\\
             & $10101111_2$\\
             & $= 1 \cdot -128 + 1 \cdot 32 + 1 \cdot 8 + 1 \cdot 4 + 1 \cdot 2 + 1 \cdot 1 = -81$
    \end{tabular}
  \end{example}
\end{frame}

\subsection{Two's Complement Arithmetic}

\begin{frame}{Why do we go to all of this trouble?}
  \begin{block}{Key Concept}
    Two's complement addition and subtraction requires the same procedure (and there the same circuits) as unsigned addition and subtraction.
  \end{block}
  \begin{example}
    \begin{columns}
      \begin{column}{5cm}
        \begin{center}
          \begin{tabular}{rcrl}
              &  5 &   & $0101_2$\\
            + & -7 & + & $1001_2$\\
            \hline
              & -2 &   & $1110_2$
          \end{tabular}
        \end{center}
      \end{column}
      \begin{column}{5cm}
        \begin{center}
          \begin{tabular}{rcrr}
              & -3 &   & $1101_2$\\
            + & -2 & + & $1110_2$\\
            \hline
              & -5 &   & $11011_2$
          \end{tabular}
        \end{center}
      \end{column}
    \end{columns}
  \end{example}
\end{frame}

\begin{itemize}
  \item Note that we ignore the carry-out.
  \item We can ignore it unless there is an overflow.
\end{itemize}

\begin{frame}{Handling addition overflow}
  \begin{itemize}
    \item Since we're dealing with a finite set of numbers (say 0-7), we need to account for overflow.
    \item The addition of two numbers with like signs may produce a number which cannot be represented.
  \end{itemize}
  \begin{block}{Overflow detection (addition)}
      Addition overflow occurs if:
    \begin{itemize}
      \item the addends' signs are the same, but the sum's sign is different
      \item if $c_{in} \neq c_{out}$ for the MSB
    \end{itemize}
  \end{block}
\end{frame}

\begin{frame}{Addition overflow examples}
  \begin{columns}
    \begin{column}{5cm}
      \begin{center}
        \begin{tabular}{rcrl}
          C &    &   & $1110$\\
            & 5  &   & $0101_2$\\
          + & 7  & + & $0111_2$\\
          \hline
            & 12 &   & $1100_2$
        \end{tabular}
      \end{center}
    \end{column}
    \begin{column}{5cm}
      \begin{center}
        \begin{tabular}{rcrr}
          C &    &   & $10000_{\text{  }}$\\
            & -3 &   & $1101_2$\\
          + & -6 & + & $1010_2$\\
          \hline
            & -9 &   & $10111_2$
        \end{tabular}
      \end{center}
    \end{column}
  \end{columns}
\end{frame}

\begin{frame}{Two's complement subtraction}
  Although two's complement subtraction is possible, usually we add the two's complement of the subtrahend to the minuend.
  \begin{block}{Two's complement subtraction procedure}
    Perform a bit complement of the subtrahend, then add the minuend and the subtrahend with a carry-in of 1.
  \end{block}
  \begin{example}
    \begin{center}
      \begin{tabular}{rcrrrl}
        C &    &   &          &   & $0011$\\
          & 5  &   & $0101_2$ &   & $0101_2$\\
        - & 7  & - & $0111_2$ & + & $1000_2$\\
        \hline
          & -2 &   &          &   & $1110_2$
      \end{tabular}
    \end{center}
  \end{example}
\end{frame}

\section {Number and Character Codes}

\subsection{Binary Coded Decimal}

\begin{frame}{Binary Coded Decimal (BCD)}
  \begin{definition}
    \alert{Binary Coded Decimal} (BCD) is a number encoding where each decimal digit is represented by its four bit binary equivalent.
  \end{definition}
  \begin{example}
    $$3 = 0011_{\text{BCD}}$$
    $$9 = 1001_{\text{BCD}}$$
    $$39 = 00111001_{\text{BCD}}$$
  \end{example}
\end{frame}

Microprocessor do not use BCD too often internally, but some peripheral devices, such as seven segment LEDs do use BCD.

\subsection{ASCII}

\begin{frame}{American Standard Code for Information Interchange (ASCII)}
  \begin{definition}
    \alert{ASCII} is a code which uses seven bits to represent the digits 0-9, letters A-Z and a-z, special characters, and control characters.
  \end{definition}
  \begin{itemize}
    \item Table 2-11 in the text on page 54 contains a table of ASCII values.
    \item The website asciitable.com contains another table that is formatted a bit differently.
    \item A number of other codes are popular, most notable, many variations of UNICODE, which can be used to represent non-English characters.
  \end{itemize}
\end{frame}

\end{document}
