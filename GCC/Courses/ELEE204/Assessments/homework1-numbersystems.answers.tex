%Created with command:
%"/home/josh/Teaching/trunk/Utilities/makeexam" "Homework 1 - Number Systems" "Attempt the following problems.  Please show all of your work.  You may work with others however make sure that all work you submit is yours.  For additional practice you may want to try problems 2.1-2.10 and 2.12 in the text." "../NumberSystems/Assessments/convert_binary_decimal_hw.tex" "../NumberSystems/Assessments/convert_hex_decimal_hw.tex" "../NumberSystems/Assessments/convert_hex_oct_bin_hw.tex" "../NumberSystems/Assessments/binary_add_hw.tex" "../NumberSystems/Assessments/binary_sub_hw.tex" "../NumberSystems/Assessments/twos_complement_hw.tex" "../NumberSystems/Assessments/twos_complement_arithmetic_hw.tex"
\documentclass{article}
\usepackage[T1]{fontenc}
\usepackage{arev}
\usepackage{longtable}
\usepackage[hmargin=2cm,vmargin=2cm]{geometry}
\usepackage{graphicx}
\setlength{\parindent}{0pt}
\title{Homework 1 - Number Systems}
\date{}
\begin{document}
\maketitle
Attempt the following problems.  Please show all of your work.  You may work with others, however make sure that all work you submit is yours.  For additional practice you may want to try problems 2.1-2.10 and 2.12 in the text. (26 points total)
\begin{longtable}[l]{rp{17cm}}
%file: ../NumberSystems/Assessments/convert_binary_decimal_hw.tex
1.&\begin{minipage}[t]{\linewidth}(2 pt) Convert the following binary numbers to decimal. \\
\\
$01001101_2$ \\
$1110.11_2$ \\

Solution: \\
\\
$01001101_2 = 1 \cdot 2^6 + 1 \cdot 2^3 + 1 \cdot 2^2 + 1 \cdot 2^0 = 64 + 8 + 4 + 1 = 77$ \\
$1110.11_2 = 1 \cdot 2^3 + 1 \cdot 2^2 + 1 \cdot 2^1 + 1 \cdot 2^{-1} + 1 \cdot 2^{-2} = 8 + 4 + 2 + 0.5 + 0.25 = 14.75$ \\
\end{minipage}\\
\medskip
%file: ../NumberSystems/Assessments/convert_hex_decimal_hw.tex
2.&\begin{minipage}[t]{\linewidth}(2 pt) Convert the following hexadecimal numbers to decimal. \\
\\
$\textrm{D}281_{16}$ \\
$3.141_{16}$ \\

Solution: \\
\\
$\textrm{D}281_{16} = 13 \cdot 16^3 + 2 \cdot 16^2 + 8 \cdot 16^1 + 1 \cdot 16^0 = 13 \cdot 4096 + 2 \cdot 256 + 8 \cdot 16 + 1 \cdot 1 = 53889$ \\
$3.141_{16} = 3 \cdot 16^0 + 1 \cdot 16^{-1} + 4 \cdot 16^{-2} + 1 \cdot 16^{-3}$\\
$= 3 \cdot 1 + 1 \cdot 0.0625 + 4 \cdot 0.00390625 + 1 \cdot 0.000244140625 \approx 3.0783$ \\
\\
\end{minipage}\\
\medskip
%file: ../NumberSystems/Assessments/convert_hex_oct_bin_hw.tex
3.&\begin{minipage}[t]{\linewidth}(4 pt) Perform the following number system conversions. \\
\\
$00101011_2 = ?_{16}$ \\
$11010011_2 = ?_8$ \\
$163_8 = ?_2$ \\
$9\textrm{A}8\textrm{C} = ?_2$ \\

Solution: \\
\\
$00101011_2 = 0010_2 1011_2 = 2\textrm{B}_{16}$ \\
$11010011_2 = 011_2 010_2 011_2 = 323_8$ \\
$163_8 = 001_2 110_2 011_2 = 1110011_2$ \\
$9\textrm{A}8\textrm{C} = 1001_2 1010_2 1000_2 1100_2 = 1001101010001100_2$ \\
\\
\end{minipage}\\
\medskip
%file: ../NumberSystems/Assessments/binary_add_hw.tex
4.&\begin{minipage}[t]{\linewidth}(2 pt) Perform the following binary addtions, showing all carries. \\
\\
$01110010_2 + 10110011_2$ \\
$01110010_2 + 10110011_2$ with $c_{in}=1$\\

Solution: \\
\\
\begin{tabular}{cccccccccc}
  C & 1 & 1 & 1 & 1 & 0 & 0 & 1 & 0 & 0 \\
    &   & 0 & 1 & 1 & 1 & 0 & 0 & 1 & 0 \\
    & + & 1 & 0 & 1 & 1 & 0 & 0 & 1 & 1 \\
  \hline
    & 1 & 0 & 0 & 1 & 0 & 0 & 1 & 0 & 1 \\
\end{tabular} \\
\\
\begin{tabular}{cccccccccc}
  C & 1 & 1 & 1 & 1 & 0 & 0 & 1 & 1 & 1 \\
    &   & 0 & 1 & 1 & 1 & 0 & 0 & 1 & 0 \\
    & + & 1 & 0 & 1 & 1 & 0 & 0 & 1 & 1 \\
  \hline
    & 1 & 0 & 0 & 1 & 0 & 0 & 1 & 1 & 0 \\
\end{tabular} \\
\end{minipage}\\
\medskip
%file: ../NumberSystems/Assessments/binary_sub_hw.tex
5.&\begin{minipage}[t]{\linewidth}(2 pt) Perform the following binary subtractions, showing all borrows. \\
\\
$10110010_2 - 00100011_2$ \\
$11110011_2 - 10101011_2$\\

Solution: \\
\\
\begin{tabular}{cccccccccc}
    &   &   &   &   &   &   &   & 11& 10\\
  B &   & 0 & 0 & 0 & 1 & 1 & 1 & 1 & 0 \\
    &   & 1 & 0 & 1 & 1 & 0 & 0 & 1 & 0 \\
    & - & 0 & 0 & 1 & 0 & 0 & 0 & 1 & 1 \\
  \hline
    &   & 1 & 0 & 0 & 0 & 1 & 1 & 1 & 1 \\
\end{tabular} \\
\\
\begin{tabular}{cccccccccc}
    &   &   &   &   &   & 10&   &   &   \\
  B &   & 0 & 0 & 0 & 1 & 0 & 0 & 0 & 0 \\
    &   & 1 & 1 & 1 & 1 & 0 & 0 & 1 & 1 \\
    & - & 1 & 0 & 1 & 0 & 1 & 0 & 1 & 1 \\
  \hline
    &   & 0 & 1 & 0 & 0 & 1 & 0 & 0 & 0 \\
\end{tabular} \\
\end{minipage}\\
\medskip
%file: ../NumberSystems/Assessments/twos_complement_hw.tex
6.&\begin{minipage}[t]{\linewidth}(2 pt) Find the eight bit two's complement representation of each of the following numbers. \\
\\
$47$ \\
$-93$ \\

Solution: \\
\\
$47 = 00101111_2$\\
\\
\begin{tabular}{rl}
  $93 =$ & $01011101_2$\\
         & $\Downarrow \textrm{complement}$\\
         & $10100010_2$\\
     $+$ & $00000001_2$\\
         & $10100011_2$\\
         & $= 1 \cdot -128 + 1 \cdot 32 + 1 \cdot 2 + 1 \cdot 1 = -93$
\end{tabular}\\
\end{minipage}\\
\medskip
%file: ../NumberSystems/Assessments/twos_complement_arithmetic_hw.tex
7.&\begin{minipage}[t]{\linewidth}(12 pt) Perform the following binary arithmetic using the four bit two's complement representation, showing all carries.  For each case, determine if overflow occurs. If it does, state why it occurs.\\
\\
\\
$3 + 2$\\
$-1 + 1$\\
$-7 + -2$\\
$6 - -6$\\

Solution: \\
\\
$3 = 0011_2$\\
$2 = 0010_2$\\
\\
\begin{tabular}{cccccc}
  C &   & 0 & 1 & 0 & 0 \\
    &   & 0 & 0 & 1 & 1 \\
    & + & 0 & 0 & 1 & 0 \\
  \hline
    &   & 0 & 1 & 0 & 1 \\
\end{tabular} \\
\\
Overflow does not occur.\\
\\
$-1 = 1111_2$\\
$1 = 0001_2$\\
\\
\begin{tabular}{cccccc}
  C & 1 & 1 & 1 & 1 & 0 \\
    &   & 1 & 1 & 1 & 1 \\
    & + & 0 & 0 & 0 & 1 \\
  \hline
    &   & 0 & 0 & 0 & 0 \\
\end{tabular} \\
\\
Overflow does not occur.\\
\\
$-7 = 1001_2$\\
$-2 = 1110_2$\\
\\
\begin{tabular}{cccccc}
  C & 1 & 0 & 0 & 0 & 0 \\
    &   & 1 & 0 & 0 & 1 \\
    & + & 1 & 1 & 1 & 0 \\
  \hline
    &   & 0 & 1 & 1 & 1 \\
\end{tabular} \\
\\
Overflow occurs because $c_{in} \neq c_{out}$ for MSB.\\
\\
$6 = 0110_2$\\
$-6 = 1010_2$\\
Note that $-6$ complement is $0101_2$ and remember to use $c_{in}=1$.\\
\\
\begin{tabular}{cccccc}
  C &   & 1 & 1 & 1 & 1 \\
    &   & 0 & 1 & 1 & 0 \\
    & + & 0 & 1 & 0 & 1 \\
  \hline
    &   & 1 & 1 & 0 & 0 \\
\end{tabular} \\
\\
Overflow occurs because $c_{in} \neq c_{out}$ for MSB.\\
\end{minipage}\\
\medskip
\end{longtable}
\end{document}