%Created with command:
%"/home/josh/Teaching/trunk/Utilities/makeexam" "Homework 8 - State Machines in VHDL" "Please complete the problem.  As usual you may work together.  However make sure the work that you submit is yours." "../StateMachines/Assessments/counter.tex"
\documentclass{article}
\usepackage[T1]{fontenc}
\usepackage{arev}
\usepackage{longtable}
\usepackage[hmargin=2cm,vmargin=2cm]{geometry}
\usepackage{graphicx}
\usepackage{listings}
\setlength{\parindent}{0pt}
\title{Homework 8 - State Machines in VHDL}
\date{}
\begin{document}
\maketitle
Please complete the problem.  As usual you may work together.  However make sure the work that you submit is yours. (15 points total)
\begin{longtable}[l]{rp{17cm}}
%file: ../StateMachines/Assessments/counter.tex
1.&\begin{minipage}[t]{\linewidth}(15 pt) Design the archiecture for a two bit binary counter in VHDL.  Use the following entity:
\lstset{language=VHDL}
\begin{lstlisting}
entity counter is
    port (clock, A, reset: in std_logic;
          Z: out std_logic);
end counter;
\end{lstlisting}
Note that all of the signals are active high.  This counter should count four consecutive clock ticks when A is asserted.  When the fourth clock tick occurs, the output Z should be asserted.  When the fifth clock tick occurs, the output Z should become zero, and remain zero until four consecutive clock ticks occur with A asserted.  If the reset intput is asserted, the next clock tick will cause the circuit to begin counting at zero.\\ \\
Note: You may use any method of design to complete this assignment, however I would recommend using direct VHDL design.\\ \\
I will provide a test bench that will be used to test your VHDL code.

\vspace{8cm
}
\end{minipage}\\
\medskip
\end{longtable}
\end{document}