%Created with command:
%"/home/josh/Teaching/trunk/Utilities/makeexam" "Homework 7 - Flip-flops and State Machines" "Please complete the problems below.  As usual you may work together.  However make sure the work that you submit is yours." "../LatchesAndFlipFlops/Assessments/wakerly_7_7.tex" "../StateMachines/Assessments/wakerly_7_18.tex" "../StateMachines/Assessments/wakerly_7_44.tex"
\documentclass{article}
\usepackage[T1]{fontenc}
\usepackage{arev}
\usepackage{longtable}
\usepackage[hmargin=2cm,vmargin=2cm]{geometry}
\usepackage{graphicx}
\usepackage{listings}
\setlength{\parindent}{0pt}
\title{Homework 7 - Flip-flops and State Machines}
\date{}
\begin{document}
\maketitle
Please complete the problems below.  As usual you may work together.  However make sure the work that you submit is yours. (24 points total)
\begin{longtable}[l]{rp{17cm}}
%file: ../LatchesAndFlipFlops/Assessments/wakerly_7_7.tex
1.&\begin{minipage}[t]{\linewidth}(4 pt) Do problem 7.7 in the text.  Your answer should be a logic diagram that includes any combinational gates and one or more T flip-flops to implement the J-K flip-flop. \\ \\

\vspace{8cm
}
\end{minipage}\\
\medskip
%file: ../StateMachines/Assessments/wakerly_7_18.tex
2.&\begin{minipage}[t]{\linewidth}(10 pt) Do problem 7.18 in the text.  Please explicitly show each of the seven steps in the procedure for state machine analysis that we discussed in class. \\ \\

\vspace{12cm
}
\end{minipage}\\
\medskip
%file: ../StateMachines/Assessments/wakerly_7_44.tex
3.&\begin{minipage}[t]{\linewidth}(10 pt) Do problem 7.44 in the text. \\ \\

\vspace{12cm
}
\end{minipage}\\
\medskip
\end{longtable}
\end{document}