%Created with command:
%"/home/josh/Teaching/trunk/Utilities/makeexam" "Homework 4 - VHDL" "Please complete the problem below.  As usual you may work together and use other resources including those in the Internet for reference.  However make sure the work that you submit is yours.  At a minimum your program should compile and run." "../VHDL/Assessments/simple_calculator.tex"
\documentclass{article}
\usepackage[T1]{fontenc}
\usepackage{arev}
\usepackage{longtable}
\usepackage[hmargin=2cm,vmargin=2cm]{geometry}
\usepackage{graphicx}
\usepackage{listings}
\setlength{\parindent}{0pt}
\title{Homework 4 - VHDL}
\date{}
\begin{document}
\maketitle
Please complete the problem below.  As usual you may work together and use other resources including those in the Internet for reference.  However make sure the work that you submit is yours.  At a minimum your program should compile and run. (12 points total)
\begin{longtable}[l]{rp{17cm}}
%file: ../VHDL/Assessments/simple_calculator.tex
1.&\begin{minipage}[t]{\linewidth}(12 pt) Using VHDL, write a four function calculcator (+, -, *, /) that operates on unsigned eight bit binary numbers.  Also, write a test bench (with proper \texttt{assert} statements) to verify that your calculator program works correctly.  Please email me your code by the beginning of class on the due date.  Hint: This assignment can be completed by creating only four enitities and one other function.  Do all of your math with integers.

Solution: \\ \\
\lstinputlisting{../VHDL/Assessments/calculator.vhd}
\lstinputlisting{%base_path%conversion.vhd}
\end{minipage}\\
\medskip
\end{longtable}
\end{document}