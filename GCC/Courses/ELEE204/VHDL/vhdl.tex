\title{VHDL}
\begin{document}
\section{VHDL Concepts}

\begin{frame}{VHDL History}
  \begin{itemize}
    \item VHDL was developed by the DoD and IEEE.
    \item It was standardized in 1987, and updated in 1993 and again in 2002.
    \item It was based on Pascal and Ada.
  \end{itemize}
\end{frame}

\begin{frame}{Features}
  \begin{itemize}
    \item VHDL is strongly typed.
    \item VHDL designs are hierarchical.
    \item It provides a well defined separation between \alert{interface} and \alert{implementation}.
    \item VHDL can model concurrency, timing, and clocking as well as synchronous and asynchronous operations.
    \item VHDL can be used to simulate the behavior of a circuit.
  \end{itemize}
\end{frame}

\begin{frame}{Interface and implementation}
  \begin{block}{Key Engineering Concept}
    Whenever possible, separate the interface of a component (how the outside world sees it), from its implementation (how it accomplishes its task).
  \end{block}
  \begin{itemize}
     \item I like to refer to this as the messy room principle.
     \item This will often save you in your career (bosses love it).
     \item Don't overkill, too much separation is just as bad - it adds unnecessary overhead and confusion.
  \end{itemize}
\end{frame}

\section{VHDL Program Structure}

\begin{frame}{Interface and implementation}
  VHDL accomplishes this separation by separating each component into two parts:
  \begin{itemize}
    \item A VHDL \alert{entity} describes a component's inputs and outputs (its interface).
    \item A VHDL \alert{architecture} describes a component's behavior (its implementation).
  \end{itemize}
  Any given architecture can use another component by instantiating its entity.
\end{frame}

\subsection{Entity Definitions}

\begin{frame}{Entity description}
  An entity consists of the following:
  \begin{itemize}
    \item One entity name
    \item One or more signal names, which consist of:
      \begin{itemize}
        \item One mode, which must be one of:
          \begin{itemize}
            \item in
            \item out
            \item buffer
            \item inout
          \end{itemize}
        \item One signal type which can be built in or user defined
      \end{itemize}
  \end{itemize}
\end{frame}

\begin{frame}{Entity example}
  \begin{example}
    Write the entity definition for a two input AND gate.
  \end{example}
\end{frame}

\subsection{Architecture Definitions}

\begin{frame}{Architecture description}
  An architecture consists of the following:
  \begin{itemize}
    \item One architecture name
    \item Zero or more type declarations
    \item Zero or more signal declarations
    \item Zero or more constant declarations
    \item Zero or more function declarations
    \item Zero or more procedure declarations
    \item Zero or more component declarations
    \item One or more concurrent statements that define how the input signals affect the output signals
  \end{itemize}
\end{frame}

\begin{frame}{Architecture example}
  \begin{example}
    Write the architecture definition for a two input AND gate with the previously created entity definition.
  \end{example}
\end{frame}

\section{VHDL Types, Constants, and Arrays}
\subsection{Types}

\begin{frame}{VHDL Types}
  VHDL has a few predefined types, including
  \begin{itemize}
    \item integer (at least $-2^{31}+1$ through $2^{31}-1$)
    \item character (basically ASCII)
    \item boolean (true or false)
  \end{itemize}
  VHDL also allows user defined types, which are usually enumerations.  The most important user defined type is STD\_ULOGIC.
\end{frame}

\subsection{Constants}

\begin{frame}{Constants}
  Constants in VHDL improve readability of code.  They consist of a name, a type and a value.
  \begin{itemize}
    \item constant WORD\_SIZE: integer := 16;
    \item constant DOUBLE\_WORD\_SIZE: integer := WORD\_SIZE*2;
    \item constant FIRST\_LETTER: character := 'A';
  \end{itemize}
\end{frame}

\subsection{Arrays}

\begin{frame}{Arrays}
  Arrays in VHDL represent a group of variables.
  \begin{itemize}
    \item type days\_in\_february is array (1 to 28) of integer;
    \item type DOUBLE\_WORD is array (DOUBLE\_WORD\_SIZE-1 downto 0) of STD\_ULOGIC;
    \item type STD\_LOGIC\_VECTOR is array (natural range <>) of STD\_LOGIC;
  \end{itemize}
  Arrays may or may not specify their number of elements in their type definition.\\
  Arrays can be set in a number of ways:
  \begin{itemize}
    \item byte := ('0','0','1','1','0','0','0','0');
    \item byte := (2=$>$'1',3=$>$'1', others=$>$'0');
    \item byte := "001100000";
  \end{itemize}
\end{frame}

\section{VHDL Functions and Packages}

\begin{frame}{VHDL functions and procedures}
  \begin{definition}
    A VHDL \alert{function} is a set of statements that execute sequentially and produce some return value.  A VHDL \alert{procedure} is like a function, but it does not return a value.
  \end{definition}
  \begin{itemize}
    \item A function must specify its \alert{parameters}, the signal values that it accepts as input.
    \item A function must specify its return type.
  \end{itemize}
  \begin{block}{Operator Overloading}
    VHDL allows operators like +, -, and, or, etc. to be overloaded, as in other languages (like C++).
  \end{block}
\end{frame}

\begin{itemize}
  \item Show AND gate function example (and\_gate\_function.vhd).
  \item Show the operator overloading AND gate example (and\_gate\_operator.vhd)
\end{itemize}

\begin{frame}{VHDL libraries}
  \begin{definition}
    A VHDL compilers stores information used for a project in a \alert{library}.  Each project has an implicit library, usually called ``work''.
  \end{definition}
  \begin{itemize}
    \item The clause \texttt{library ieee;} causes the ieee library to be available to the file.
    \item Each VHDL file has an implicit \texttt{library work;} clause.
  \end{itemize}
\end{frame}

\begin{frame}{VHDL packages}
  \begin{definition}
    A \alert{package} is a set of signals, types, constants, functions, procedures, or component declarations that can be used in other VHDL files.
  \end{definition}
  \begin{itemize}
    \item Packages can be included with the use clause, as in \texttt{use ieee.std\_logic\_1164.all;}.
    \item Parts of packages can be selective included as needed, as in \texttt{use ieee.std\_logic\_1164.std\_ulogic;}.
  \end{itemize}
\end{frame}

Show AND gate function with std\_logic.

\section{VHDL Design}

\begin{frame}{Concurrent statements}
  \begin{itemize}
    \item Remember, in VHDL, each statement is concurrent.  That is, each statement executes logically at the same time as all other statements.
    \item This means that a change to a value of a signal in a later statement may effect that signals value in a previous statements.
    \item A statement as seemingly innocuous as \texttt{X <= not X} is actually an infinite loop.
  \end{itemize}
  Note to worry though, we can in fact take advantage of this concurrency to perform intuitive design.
\end{frame}

\subsection{Structural Design}

\begin{frame}{Structural design}
  \begin{itemize}
    \item Often we think of digital logic circuits as just that - circuits, where everything happens at once.
    \item This is natural concurrency.  We apply the inputs, and the output is immediate available.
    \item So we can design VHDL this way, like the roommate control circuit.
  \end{itemize}
\end{frame}

Do the roommate control circuit structural example.

\subsection{Dataflow Design}

\begin{frame}{Dataflow design}
  Instead of considering the actual logic gates of a circuit, we can think about the signals within the circuit.
  \begin{block}{Conditional signal assignment}
    VHDL allows for signals to be assigned conditionally, like this \texttt{HUNGRY <= '1' when not FULL(me);}.
  \end{block}
  \begin{block}{Explicit signal selection}
    We can even select the signal output by examining all of the possible inputs:\\
    \texttt{with NUMBER select\\ \t EVEN <= '1' when "00" | "10", '0' when others;}
  \end{block}
\end{frame}

Do the roommate control circuit dataflow example.

\subsection{Behavioral Design}

\begin{frame}{The process statement}
  \begin{definition}
    The VHDL \alert{process} statements defines a list of sequential statements that execute when a signal specified in the \alert{sensitivity list} of the process changes.
  \end{definition}
  \begin{itemize}
    \item When a signal that a given process cares about changes, the statements in the process are executed sequentially.
    \item After all statements are executed, the signals assigned values in the process are updated and the circuit execution continues at the next time step.
    \item A process may contain \alert{variables} which can be used to compute intermediate values that are not tied directly to signals.
  \end{itemize}
\end{frame}

Do the roommate control circuit behavioral example.

\section{VHDL Control Structures}

\begin{frame}{VHDL if statement}
  VHDL provides an if/else if /else statement, which is spelled like this:
  \begin{block}{VHDL if/else if/else}
    \texttt{if homework=due\_next\_week then ACTION <= 'relax';\\elsif homework=due\_tomorrow then ACTION <= 'doit';\\else ACTION <='panic';\\end if;}
  \end{block}
\end{frame}

\begin{frame}{VHDL case statement}
  VHDL provides a case statement, which is often more readable than an if/else chain.  It can also be easier for a synthesizer to implement.
  \begin{block}{VHDL case}
    \texttt{case homework is\\ \hspace{24pt} when due\_next\_week => ACTION <= 'relax';\\ \hspace{24pt} when due\_tomorrow => ACTION <= 'doit';\\ \hspace{24pt} when others => ACTION <='panic';\\end case;}
  \end{block}
\end{frame}

\begin{frame}{VHDL for loop}
  VHDL provides a for loop, which is spelled like this:
  \begin{block}{VHDL for Loop (Gauss in kindergarten)}
    \texttt{for i in 1 to 100 loop\\ \hspace{24pt} sum = sum + i;\\ end loop;}
  \end{block}
\end{frame}

\begin{frame}{VHDL while loop}
  VHDL also provides a while loop, which is spelled like this:
  \begin{block}{VHDL while Loop (Gauss in kindergarten again)}
    \texttt{i=1;\\ while i <= 100 loop\\ \hspace{24pt} sum = sum + i;\\ end loop;}
  \end{block}
\end{frame}

\end{document}
