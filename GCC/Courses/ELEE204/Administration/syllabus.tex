\documentclass{article}
\title{ELEE 204 - Digital Logic Design\\Course Syllabus}
\date{}
\usepackage[T1]{fontenc}
\usepackage{arev}
\usepackage{array}
\usepackage[hmargin=2cm,vmargin=2cm]{geometry}
\usepackage{fancyhdr}
\pagestyle{fancy}
\fancyhead{}
\fancyfoot{}
\renewcommand{\headrulewidth}{0pt}
\fancyhead[LO,LE]{ELEE204}
\fancyhead[CO,CE]{Digital Logic Design}
\fancyhead[RO,RE]{Spring 2009}
\cfoot{\thepage}
\begin{document}
\maketitle
\thispagestyle{fancy}
\section{Course Details}
\subsection{Catalog Description}
An introduction to digital circuit analysis and design methods. Combinational circuit topics include the use of Boolean algebra, map minimization methods, and circuit implementation with logic gates and standard integrated circuits. Sequential circuit design is explored, and implementation with flip flops and standard integrated circuits is investigated. Programmable logic implementation of both combinational and sequential circuits is introduced.  A group design project is required.
\subsection{Prerequisites}
This course has no prerequisites.
\subsection{Course Overview}
An introduction to digital logic design and hardware engineering with an emphasis on practical design techniques and circuit implementation, intended to serve as a prerequisite for upper-division courses in the computer engineering curriculum.  Through the completion of homework exercises and laboratory experiments (from ELEE 252), students will learn how to design digital circuits, specify their functionality using a hardware description language, and realize as well as test these circuits in lab using programmable logic devices.
\subsection{Meeting}
TR 8:00 - 9:15 AM Hoyt Hall 112
\subsection{Text}
\emph{Digital Design: Principles and Practices, 4th ed.}, by Wakerly, Pearson Prentice Hall, 2006, ISBN 0-13-186389-4
\subsection{Course Website}
my.gcc.edu
\subsection{Faculty Details}
Mr. Josh Peterson, Office Hoyt Hall 101 jmpeterson@gcc.edu, 724-825-8881
\subsection{Office Hours}
TR 9:15 - 9:45 and by appointment
\section{ABET Program Outcomes}
By the conclusion of the course, students must demonstrate:
\begin{itemize}
\item[(c)] An ability to design a system, component, or process to meet desired needs.
\begin{itemize}
\item[-]Much of the class is devoted to designing logic circuits to solve specified problems.  Students will learn to combine gates and higher function chips to create combinational and sequential logic.  The latter part of the class will focus on finite state machines.
\end{itemize}
 \item[(e)] An ability to identify, formulate, and solve engineering problems.
 \begin{itemize}
 \item[-]Digital logic is useful for solving many practical problems, so real-life examples will be used throughout the class.  The project will have some freedom for the student to identify a realistic problem.
 \end{itemize}
  \item[(g)] An ability to communicate effectively, including the ability to write clearly and cohesively about technical subjects, communicate mathematical analyses in a comprehensible form, and orally communicate on technical subjects with people at all different levels of technical ability.
  \begin{itemize}
  \item[-]Digital logic designs can be specified in multiple forms.  Students will practice converting between forms and selecting the best form for a particular need.  Students will need to communicate complicated designs in a clear and effective fashion on all assigned work.
  \end{itemize}
   \item[(k)] An ability to use the techniques, skills, and modern engineering tools necessary for engineering practice.
   \begin{itemize}
   \item[-]Logic design layout will be discussed.  Students will specify some designs using a VHDL simulation environment
   \end{itemize}
\end{itemize}
\section{Course Outcomes}
Upon completion of the course, students will:
\begin{enumerate}
  \item Add and subtract using binary numbers. (supports e and k)\\
  \textbf{Assessment} - in-class exercises, homework problems and test questions
  \item Transform logic functions between Boolean algebra, truth tables and circuit diagrams forms. (supports e and g) \\
  \textbf{Assessment} - in-class exercises, homework problems and test questions
  \item Generate minimal solutions to logic functions.  (supports e and k) \\
  \textbf{Assessment} - in-class exercises, homework problems and test questions
  \item Design circuits using combinational Integrated Circuits (ICs) such as decoders, MUXes, comparators and adders.  (supports c and e) \\
  \textbf{Assessment} - homework problems, project and test questions
  \item Design circuits using sequential ICs such as latches, flip-flops, counters and shift registers. (supports c and e) \\
  \textbf{Assessment} - homework problems, project and test questions
  \item Analyze and design solutions using Finite State Machines. (supports c, e and k) \\
  \textbf{Assessment} - homework problems, project and test questions
\end{enumerate}
\section{Methodology}
\subsection{Teaching Methods}
This course will primarily consist of class lectures.  We will do some in-class exercises to practice using new techniques and components.

\subsection{Assessment Procedures}
There will be (ungraded) in-class exercises, regular homework, regular quizzes, a group project, 2 semester exams and one final exam.

\subsection{Grading}
  The following assessments will be given.  Grades will determined using the following scale: 100-98 A+, 97-92 A, 91-90 A-, 89-88 B+, 87-82 B, 81-80 B-, etc.\\ \\
\begin{tabular}{ll}
  2 Semseter exams (15\% each) & 30\% \\
  1 Final exam                 & 20\% \\
  1 Group project              & 10\% \\
  4 Quizzes (5\% each)         & 20\% \\
  10 Homeworks (2\% each)      & 20\% \\
\end{tabular}

\section {Policies}
\subsection{Academic Integrity}
Each student must read and understand the Honesty in Learning and Plagiarism policies given in the Crimson.
\\
\begin{itemize}
  \item As a general rule, all work submitted for grading must be the student's own work.  Students are allowed to generally discuss issues and broad solutions related to a particular problem, but students may not share, write or examine another student's actual answers.
  \item In regard to homework, students may discuss the problems (what is being asked for), use appropriate material from class lectures or the textbook or acceptable other sources.  Students, however, may not share answers or the specifics of how to answer the question.
  \item Use of material from previous classes, solution manuals, material from the Internet or other sources (e.g. parents, siblings, friends, etc.) that directly bears on the answer is strictly prohibited.
\end{itemize}
At the professor's discretion, students may be asked to sign a statement that they have abided by the College's Academic Integrity policy and its application to this class.  This statement may appear on homework, tests or projects.
\\ \\
When in doubt, consult the course professor before doing something that may result in violation of the college's Academic Integrity policy.
\\ \\
Application to this course: The first violation of the Honesty in Learning or Plagiarism policy will result in a 0 for the relevant assignment and, as per college policy, the Provost's office will be notified.  A second violation will result in failing the course and other penalties at the discretion of the College administration.

\subsection{Computer Use Policy}
Students are welcome to use their computers for note-taking or other class-relevant activities. The manner in which you use your computer in class is considered a matter of honor and professionalism. Students are to adhere to the following guidelines:
\\
\begin{itemize}
  \item The student computer should not be connected to the network (wireless or wired) unless instructor-initiated classroom activities require the network.
  \item Computer use must be for note-taking or other class-relevant activities.
  \item Use of the computer must be subtle and non-distracting to classmates and the instructor.
\end{itemize}

Inappropriate use of a computer in the classroom may be viewed as being disrespectful to the instructor, is often distracting to other students, and is unprofessional. Examples (not a comprehensive list) of inappropriate activities include:
\\
\begin{itemize}
  \item E-mailing
  \item Instant Messaging
  \item Surfing the Web
  \item Working on projects/assignments for other classes (or the current class unless directed to do so by the instructor)
  \item Playing games
  \item Watching movies
  \item Listening to music
\end{itemize}

Judgment as to the appropriateness of student computer use is at the discretion of the instructor. The consequences for violating this policy are also at the discretion of the instructor.

\subsection{Homework Policy}
Homework is due at the beginning of class on the due date.  Assignments turned in after that but before midnight (of that day) receive a 10\% penalty.  Assignments turned in by midnight of the following day receive a 30\% penalty.  After midnight the following day, the assignment receives 0\%.

\subsection{Exams Policy}
Exams are closed book.  You are allowed to bring in 1 page (8.5x11") of notes.  I may review these note sheets during or after the test.  Unless otherwise stated, no calculators are allowed.  Most of this material is simple algebra, which you should be able to perform on your own.  The final will be cumulative.

\section{Schedule and Readings}
This is a tentative schedule for the course. \\ \\
\begin{tabular}{|>{\raggedright}m{4cm}|>{\raggedright}m{6.5cm}|>{\raggedright}m{4cm}|}
  \hline
  \textbf{Meeting Date} & \textbf{Topic} & \textbf{Text Section} \tabularnewline \hline
  Tuesday, January 20 & Introduction, number systems & 1.1-1.12, 2.1-2.3 \tabularnewline \hline
  Thursday, January 22 & Number systems and codes & 2.4-2.6, 2.10, 2.13 \tabularnewline \hline
  Tuesday, January 27 & Digital circuits - CMOS & 3.1-3.4 \tabularnewline \hline
  Thursday, January 29 & CMOS electrical characteristics & 3.5-3.6 \tabularnewline \hline
  Tuesday, Febuary 3 & CMOS design, Digital curcuits - TTL & 3.10 \tabularnewline \hline
  Thursday, Febuary 5 & Quiz 1, Combinational circuits & 4.1 \tabularnewline \hline
  Tuesday, Febuary 10 & Combinational Circuits & 4.2-4.3 \tabularnewline \hline
  Thursday, Febuary 12 & Combinational Circuits & 4.3 \tabularnewline \hline
  Tuesday, Febuary 17 & VHDL & 5.1, 5.3 \tabularnewline \hline
  Thursday, Febuary 19 & Exam 1, VHDL & 5.3 \tabularnewline \hline
  Tuesday, Febuary 24 & VHDL, Combinational design & 5.3, 6.1 \tabularnewline \hline
  Thursday, February 26 & No class & None \tabularnewline \hline
  Tuesday, March 3 & Combinational design, Circuit timing & 6.1-6.2 \tabularnewline \hline
  Thursday, March 5 & Quiz 2, Decoders, encoders & 6.4-6.5 \tabularnewline \hline
  Tuesday, March 10 & Multiplexors, XOR & 6.7-6.8\tabularnewline \hline
  Thursday, March 12 & Comparators & 6.9 \tabularnewline \hline
  Tuesday, March 17 & Math circuits & 6.10 \tabularnewline \hline
  Thursday, March 19 & Exam 2, Latches & 7.1-7.2 \tabularnewline \hline
  Tuesday, March 24 & Latches, Flip-flops & 7.1-7.2 \tabularnewline \hline
  Thursday, March 26 & State machines & 7.3-7.4 \tabularnewline \hline
  Tuesday, March 31 & State machines & 7.3-7.4 \tabularnewline \hline
  Thursday, April 2 & Quiz 3, State machines in VHDL & 7.12 \tabularnewline \hline
  Tuesday, April 7 & No class & None \tabularnewline \hline
  Thursday, April 9 & No class & None \tabularnewline \hline
  Tuesday, April 14 & Counters & 8.4 \tabularnewline \hline
  Thursday, April 16 & Shift registers & 8.5 \tabularnewline \hline
  Tuesday, April 21 & Synchronization & 8.7 \tabularnewline \hline
  Thursday, April 23 & Memory & 9.1-9.2 \tabularnewline \hline
  Tuesday, April 28 & Memory & 9.3-9.4 \tabularnewline \hline
  Thursday, April 30 & Quiz 4, Simple CPU & None \tabularnewline \hline
  Tuesday, May 5 & Simple CPU & None \tabularnewline \hline
  Monday, May 11 & Final Exam - 7:00 PM & \tabularnewline \hline
\end{tabular}
\end{document}
